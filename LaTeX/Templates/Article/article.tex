\documentclass[a4paper,,twocolumn]{article}
\usepackage[margin=0.8in, includeheadfoot]{geometry}
\usepackage[utf8]{inputenc}
\usepackage{float}
\usepackage[style=numeric,autocite = superscript,backend=biber, dateabbrev=false,isbn=false]{biblatex}
\addbibresource{bibliography.bib}
%\usepackage{CormorantGaramond}
\usepackage{titlesec}
\usepackage{hyperref}
\usepackage[dvipsnames]{xcolor}
\usepackage{fancyhdr,lastpage}
\usepackage{graphicx}
\usepackage{Zallman,lettrine}
\renewcommand\LettrineFontHook{\Zallmanfamily}
\usepackage[font={color=darkgray,small},labelfont={bf,it}]{caption}

\hypersetup{
	colorlinks,
	linkcolor={red!50!black},
	citecolor={blue!50!black},
	urlcolor={blue!80!black}
}

\titlespacing*{\section}
{0pt}{ -.1cm}{-.2cm}

% Change format of visited on' in references
\DefineBibliographyStrings{english}{
	urlseen = {Acessed on }
}
\DeclareFieldFormat{urldate}{\mkbibbrackets{\bibstring{urlseen}\thefield{urlday}\addspace\mkbibmonth{\thefield{urlmonth}}\addspace\thefield{urlyear}}
}

%Header & Footer%
\pagestyle{fancy}
\renewcommand{\headrulewidth}{0pt}
\let\oldheadrule\headrule% Copy \headrule into \oldheadrule
\renewcommand{\headrule}{\color{Gray}\oldheadrule}% Add colour to \headrule
\lhead{\color{Gray}{\footnotesize{`Gorkha Earthquake In Retrospect'}}}
\rhead{\color{Gray}{\footnotesize{BIJU ALE}}}
\cfoot{}
\rfoot{\thepage\ of {\pageref{LastPage}}}

%Include page number on title page
\fancypagestyle{plain}{
	\lhead{}
	\rhead{}
	\cfoot{}
	\rfoot{\thepage\ of {\pageref{LastPage}}}
}

\title{Gorkha Earthquake in Retrospect.  \\ \large{Lessons in Diplomacy, Preparedness, \& Morality.}\vspace{-1ex}}
\author{Biju Ale \\Harvard International Review\\ \small{\url{https://hir.harvard.edu/gorkha-earthquake/}}}
\date{Oct 02, 2020\vspace{-3ex}}

\begin{document}
	\maketitle
	\setlength{\parskip}{.5em}
	
	\lettrine{B}{iju Ale}, \textit{is an independent inter-disciplinary researcher. He was the logistics aide for DanChurchAid’s Gorkha-earthquake response project in 2015, and a former member of UN WFP’s logs cluster.}
	
	\section*{Tectonic Tragedy}
	
	Two decades ago, the British seismologist Roger Bilham expounded on the inevitable ‘next great quake’\autocite{roger_bilham_next_2015} in the Himalayas. On April 25, 2015, the earthquake\autocite{roger_bilham_subterranean_2015} measuring magnitude 7.8 (and a subsequent aftershock of magnitude 7.3 of May 12) claimed nearly 9000 lives, crumbled 600,000 houses, and injured 22,000. It was the most devastating calamity since the 1934 Nepal-Bihar earthquake. Much of the likes of Bilham’s prognosis was fulfilled; despite all the tragedy, the 2015 earthquake was not the anticipated ``big one'':  the great quake still lurks in the future, readying to unleash havoc somewhere in western Nepal. The good news is that the lessons learned from 2015 can significantly mitigate losses of the impending calamity.
	
	Within 48 hours of the tremor, the United Nations prepared its global flash appeal for Nepal. State, civil and military groups, and national and the international humanitarian organizations sprang into action in rescue missions. The early response was coordinated by the United Nations Office for the Coordination of Humanitarian Affairs under the technical cluster system. I was part\autocite{logistics_cluster_logistics_2015} of the response, with DanChurchAid (DCA)\autocite{nodhjaelp_about_2018} from May to November as DCA’s logistics aide deployed in Gorkha, coordinating our humanitarian operation with\autocite{logistics_cluster_meeting_2015} the logs cluster.
	
	This article will delineate serveral imperatives for Nepal across two domains, domestic and foreign, set in the backdrop of the earthquake. Regarding domestic affairs, I begin by assessing the socio-economic nature of the impact. Then I discuss an emergent moral callousness of nationalist urbanites ensuing from the iconization of a toppled monument. I likewise draw attention to the serious threat of casteism to aid equality and human rights in the dominant Hindu society. In the foreign affairs front, I examine myriad episodes of opportunist disaster politics, both subtle and explicit, played-out during the crisis; I  inspect the Indian response,  their journalism, and Nepal’s deflection of a British grant. Subsequently, I highlight why Nepal needs an institutional intelligentsia sensitive to humanitarian scenarios. In summary, I propose a cumulative case for the need for a cross-cutting moral underpinning that strengthens Nepal’s preparedness, risk reduction, and disaster resilience.
	
	\begin{figure}[H]
		\includegraphics[width=\linewidth]{example-image-a}
		\vspace{-4ex}
		\color{Sepia}{\caption{Labeled section of Nepal’s map with highlighted landslides triggered by the 2015 Gorkha earthquake in Nepal. An international volunteer geohazards team implemented a range of imaging and satellite technologies to generate this map. ``Insights from Earthquake in Nepal'' by NASA's Goddard Space Flight Center is licensed under Attribution 2.0 Generic (CC BY 2.0).}}
		\vspace{-4ex}
	\end{figure}
	
		
	\textit{The views expressed in this article are the author's own; they do not represent the institutions of his past or present affiliations.}
	\printbibliography
\end{document}